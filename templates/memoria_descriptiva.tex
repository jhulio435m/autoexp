\documentclass{/home/jhulio/autoexp/core/expediente}

% Datos Generales
\entidad{PROVIAS NACIONAL}
\nombreproyecto{MEJORAMIENTO DEL CAMINO VECINAL EMP. PE-1N - VICHAYITO - PENA MALA - EMP. PE-1N (MANCORA)}
\ubicacion{LOS ORGANOS Y MANCORA - TALARA - PIURA}
\cui{2434446}
\volumen{VOLUMEN III: MEMORIA DESCRIPTIVA}
\consultor{I H ASESORES Y CONSULTORES S.A.C.}

\begin{document}

\hacerportada

\tableofcontents
\newpage

\section{GENERALIDADES}
\subsection{Introducción}
El presente estudio corresponde al Expediente Técnico detallado para el mejoramiento de la transitabilidad en la zona norte de Piura.

\subsection{Antecedentes}
Se fundamenta en la necesidad de conectar los corredores logísticos turísticos de Vichayito y Máncora, bajo el marco del Plan de Acción de Transportes (PATS).

\subsection{Ubicación}
\begin{itemize}
    \item \textbf{Distrito:} Los Órganos / Máncora
    \item \textbf{Provincia:} Talara
    \item \textbf{Departamento:} Piura
\end{itemize}

\section{BASES LEGALES Y NORMATIVAS}
Para el desarrollo de los estudios, se ha tenido en cuenta la siguiente normativa sectorial de TRANSPORTES (Extraída de SEACE):

\begin{enumerate}
    \item Ley N° 30225, Ley de Contrataciones del Estado y su Reglamento.
    \item Manual de Carreteras: Diseño Geométrico (DG-2018).
    \item Resolución Ministerial N° 192-2018-VIVIENDA.
\end{enumerate}

\section{DESARROLLO DE LOS ESTUDIOS}
\subsection{Estudio de Topografía}
Se ha realizado el levantamiento mediante estación total y dron, estableciendo los hitos de control necesarios.

\subsection{Estudio de Hidrología y Drenaje}
Análisis de las alcantarillas existentes y propuesta de nuevas estructuras para el manejo de aguas pluviales.

\section{DESCRIPCIÓN DEL PROYECTO}
\subsection{Metas Físicas}
Longitud total de la vía: XX.XX Km.
Tipo de Pavimento: Tratamiento Superficial Bicapa.

\end{document}
