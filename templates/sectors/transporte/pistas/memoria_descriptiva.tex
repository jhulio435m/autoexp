\subsection*{ANTECEDENTES}
El proyecto de infraestructura vial en la localidad de {{ localidad }} surge como respuesta a la necesidad de mejorar la transitabilidad y conectividad de la zona. Actualmente, las vías presentan un estado de deterioro avanzado, dificultando el transporte de personas y productos.

\subsection*{OBJETIVOS DEL PROYECTO}
\begin{itemize}
    \item Mejorar la transitabilidad vehicular y peatonal en el sector de {{ localidad }}.
    \item Reducir los tiempos de transporte y costos operativos de los vehículos.
    \item Garantizar la seguridad vial mediante la señalización y diseño técnico adecuado.
\end{itemize}

\subsection*{DESCRIPCIÓN TÉCNICA DE LAS METAS}
El proyecto comprende la intervención en un área total estimada según el levantamiento topográfico, incluyendo:
\begin{itemize}
    \item Pavimentación flexible/rígida según diseño estructural.
    \item Construcción de veredas de concreto de resistencia f'c=175 kg/cm2.
    \item Sistema de drenaje pluvial para evacuar las aguas de escorrentía.
    \item Señalización horizontal y vertical.
\end{itemize}

\subsection*{PRESUPUESTO ESTIMADO}
El presupuesto total asciende a S/. {{ "{:,.2f}".format(monto|float) }}, el cual ha sido calculado considerando los precios vigentes en la región de {{ departamento }}.
