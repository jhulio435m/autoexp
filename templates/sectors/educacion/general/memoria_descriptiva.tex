\subsection*{ANTECEDENTES}
La Institución Educativa {{ entidad }} ubicada en la localidad de {{ localidad }}, requiere una intervención integral debido al estado de su infraestructura actual. El crecimiento de la población estudiantil ha superado la capacidad instalada, afectando las condiciones pedagógicas.

\subsection*{OBJETIVOS DEL PROYECTO}
\begin{itemize}
    \item Ampliar la infraestructura educativa para cubrir la demanda actual de {{ beneficiarios }} alumnos.
    \item Dotar de ambientes adecuados según las normas técnicas vigentes del PRONIED.
    \item Garantizar condiciones de seguridad y accesibilidad universal en el plantel.
\end{itemize}

\subsection*{DESCRIPCIÓN TÉCNICA (INFRAESTRUCTURA EDUCATIVA)}
La intervención contempla:
\begin{itemize}
    \item Construcción de aulas pedagógicas con acabados de alta resistencia.
    \item Implementación de laboratorios y salas de cómputo.
    \item Cerco perimétrico y patios de recreación techados.
    \item Servicios higiénicos diferenciados por niveles y géneros.
\end{itemize}

\subsection*{NORMATIVA APLICABLE}
El diseño se basa estrictamente en la RM 084-2019-MINEDU que establece los criterios para la elaboración de expedientes técnicos en el sector educación.
